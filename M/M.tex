\section{Kalkulation}

\subsection{Prüfungsfragen}
Erläutern Sie die einmaligen und zeitgebundenen Baustellengemeinkosten (BGKO)! Wie hoch sind ca. die BGKO und wie setzen sich diese zusammen? \newline \newline
#Grafik Kostenaufteilung \newline
#Grafik Hochbau vs Tiefbau \newline \newline
Wie können Baustellengemeinkosten umgelegt werden und warum können daraus Nachteile entstehen? \newline \newline
Umlauf auf: Preisanteile, Lohn, Mittellohn ect...\newline
Nachteile: mengenabhängig, Zeitfixiert, Verfälschung der Preisanteile \newline \newline
Sie erreichten ein Wohnhaus und wickeln den materialtransport mit einem Bauaufzug ab. Lediglich für den letzten Stock benötigen sie einen Kran - sie sieht das LV aus?\newline \newline
Sie sind Bauleiter und müssen sich Leistungspositionen selber kalkulieren - wie gehen Sie vor? Welche Unterlagen benötigen Sie for der Kalkulationsabteilung?\newline \newline
Woher bekommen Sie Priese, wenn nicht der Betriebsdatenbank vorhanden?\newline \newline
Was muss ein Kalkulat beachten?\newline\newline
Erklären sie die Vorgehensweise der Arbeitskalkulation\newline\newline
Wie erhalten sie Arbeitszeitrichtwerte und Richtwerte für die Kalkulation?\newline\newline
Was versteht man unter einem Soll-Ist-Vergleich?\newline\newline
Material, Mengen, Stunden\newline\newline
Abteilung/Zuteilung der Arbeit auf Bauleiter/Polier\newline \newline
Bauteileitung
\begin{itemize}
\item Disposition (Entscheidung über Einsatz)
\item Kontrolle und Überwachung
\item Informationstätigkeit
\item Arbeitsvorbereitung
\item Besprechung und Verwaltungsarbeit (Bautagebuch, \dots)
\end{itemize}
Polier\newline\newline
Organisation, Überwachung und Protokollierung des gewerblichen Personal- und Geräteeinsatzes anhand der vorliegenden Arbeisablauf-, Termin-, Personaleinsatz- und Gerätepläne\newline\newline
Was ist eine Bauerfolgsrechung?
\begin{itemize}
\item Periodische Feststellung des wirtschaftlichen Ergebnisses einer Baustelle
\item Gegeüberstellung von Aufwand und Ertrag von Baubeginn bis Stichtag
\end{itemize}
Veränderliche Preise - Vorteile, Nachteile, Möglichkeiten\newline
Preis, der bei Änderungen vereinbarten Preisgrundlagen unter bestimmten Voraussetzungen geändert werden kann.\newline\newline
\subsection{Beispiel Baustelleneinrichtung}
Grundstück (L=30m, B=25m)\newline
Randbedingungen (Geschlossene Bebauung, Eckgrundstück, Bohrpfähle als Sicherung)\newline
#Grafik\newline\newline
Bauwerkstyp Beschäftighte pro Kran\newline
Mauerwerksbau 15\newline
Betonbau, betonieren mittels Rohrförderung 20-30\newline
Betonbau, betonieren mit Kran 15-20\newline
Fertigteilmontage 3-5\newline\newline
#Bestimmung Auslegelänge und Höhe + Grafik dazu \newline
# Bemessung Trumdrehkran + Grafik dazu \newline
# Turmdrehkran - ÖBGL Turmdrehkran mit Laufkatze + Grafik \newline
\subsection{Bemessung der Tagesunterkünfte}
Bereich Anforderung
lichte Mindestraumhöhe : 2,30m, bei Container 2,0m\newline
Raumtemperatur in der kalten Jahreszeit: mindestens 21Grad C, Windfang am Eingang\newline
freie Bodenfläche pro Person: 0,75m2, bei Raumhöhen bis 2,30m 1,00m\newline
Tischfläche pro Person: mindestens 60cm breit und 30cm hoch\newline
Sitzplatz pro Person: 1 Sitzplan pro Person\newline
Spind pro Person: mind. 50 cm breit, 50cm tief, 180cm hoch \newline
Kochnische: muss vorhanden sein\newline
Nichtrauerschutz: geeignete Maßnahmen\newline
ca. 10 Arbeiter in 1 Container\newline\newline
Anzahl der Waschplätze: eine Waschgelegenheit pro 5 Argbeitnehmer\newline
Anzahl der Brauseplätze: eine Brauseeinrichtung pro 20 Arbeiter\newline
Aborte: eine Abortanlage pro 20 männliche bzw. 15 weibliche Arbeitnehmer\newline
Pissoirs: ein Pissoir pro 15 männliche Arbeitnehmer\newline\newline
Wasserversorgung:\newline
Trinkwasser und Waschen: Tagesunterkunft; 20-30l/Mann und \newline
Trinkwasser, Waschen und Duschen: Wohl-Schalfunterkufnt; 40-70 l/Mann und\newline
Zuschlag für längere provisorische Zuleitungen: ; 10-20 \%\newline \newline
Bemessung STrombedarf\newline
\begin{itemize}
\item Baumschinen
\item Beleuchtung
\item Bürogeräte (Computer, Kopierer\dots)\newline
\item Warmwasseraufbereitung
\end{itemize}
Bemessung Strombedarf\newline
#siehe Tabelle\newline\newline
#Baustelleneinrichtungsplan \newline

