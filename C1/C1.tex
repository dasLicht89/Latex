\section{Infrastruktur}
Vortragender: Michael Berger\newline
\subsection{Kreislauf}
Magma durch Gänge nach oben; Vulkan. (Abbildung 1.3 Skriptum)\newline
Tiefengesteine (Magmatite) \newline
Vulkanite; Plutonite\newline
werden durch Erosion zerlegt und durch Wind, Wasser, ect. verfrachtet. Durch Überlagerungsdruck können diese wieder verfestigt werden. Das sind dann die Metamorphite.\newline
Note: Festgesteine(Fels) und Lockergesteine(Boden)\newline
\newline
Erdneuzeit: Aufbau in Wien hat sich hier gebildet. Sedimente aus Wien stammen aus den Quartär(schotter) bzw. Miozön(wrn. Tegel;feinkörnige Sedimente)\newline
Mineralböden: 
\begin{itemize}
	\item nichtbindige Böden
	\item schwachbindige Böden
	\item bindige Böden
\end{itemize}
Typische Böden in Österreich:\newline
\begin{itemize}
	\item Flyschzone: verfestigtes Sedimentgestein. Sandstein,Schluffstein,Tonstein; Bietet günstige voraussetzungen für Hangrutschungen.
	\item wr. Tegel: feinkörniges, stark bindiges Sediment; gepannte Grundwässer; neigt zu Hebungen bei Lastabtragung; stark überkonsolidiert.
	\item Lösse: feine Sedimentsande. 
	\item Schotter: Kiese mit Sandbeimischungen. Von der Donau über Verfrachtung abgelagert; vorrangig abgerundetes Gestein. 
\end{itemize}
Geologie von Wien\newline\newline
Bodenklassifizierung und Bodenkennwerte\newline
Boden 3-phasen-Modell\newline
Korngrößenverteilung:\newline
Tabelle im Skriptum\newline
Kennwerte Bestimmen\newline
Bodenkennwerte:
