\section{Baurecht}

\subsection{Formelle Erfordernisse bei Bauvorhaben}

\begin{tcolorbox}[colback=lightgray!5!lightgray,colframe=lightgray!75!black,title=Bewilligung]
	\begin{itemize}
		\item Bewilligungspflichtige Bauvorhaben §60
		\item Anzeigepflichtige Bauvorhaben §62
		\item Bewilligungsfreie Bauvorhaben §62a
	\end{itemize}
\end{tcolorbox}

\begin{tcolorbox}[colback=lightgray!5!lightgray,colframe=lightgray!75!black,title=\textbf{Bewilligungsverfahren}]
	\begin{itemize}
		\item Bauverhandlung und Baubewilligung §70
		\item Vereinfachtes Bewilligungsverhahren §70a / §70b
		\item Bewilligung für Bauten vorübergehenden Bestand §71
		\item Bewilligung für Bauten lagen Bestandes §71a
		\item Sonderbaubewilligungen §71b
		\item Vorübergehende Unterbringung §71c
	\end{itemize}
\end{tcolorbox}
\begin{tcolorbox}[colback=lightgray!5!lightgray,colframe=lightgray!75!black]
Für die Erwirkung der Baubewilligung ist nach der Bauordunung (BO) der \textbf{BAUHERR} verantwortlich
\end{tcolorbox}

§61 Bewilligung von Anlagen
\begin{tcolorbox}[colback=lightgray!5!lightgray,colframe=lightgray!75!black]
Nur wenn nicht nach allem bundes- oder landesgesetzlichen Vorschriften zu Beweilligen ist
\end{tcolorbox}

§61a Besondere Regelungen für Seveso-Betriebe
\begin{tcolorbox}[colback=lightgray!5!lightgray,colframe=lightgray!75!black]
	Der Neu-, Zu-, und Umbau von Betrieben, die in den Anwendungsbereich der Richtliniee 2012/18/EU des Europäischen Parlaments und des Rates zur Änderung und anschließenden Aufhebung der Richtiline 96/82/EG des Rates fallen ("Serveso-Betriebe"), die Nutzungsänderungen zu einem Serveso-Betrieb sowie die wesentliche Änderung von solchen Betrieben bedürfen einer Bewilligung und sind so zu planen und auszufürhen, das seine erhebliche Erhöhung des Risikos oder der Folgen eines schweren Unfalls innerhalb des angemessenen Schutzabstandes eines Serveso-Betriebes, insbesondere hinsichtlich der Anzahl der betroffenen Persoinen, ausgeschlossen oder durch Setzung von sonstigen ortganisatorischen oder technischen Maßnahmen angewendet werden kann.
\end{tcolorbox}

§62 Bauanzeige
\begin{tcolorbox}[colback=lightgray!5!lightgray,colframe=lightgray!75!black]
Genügt für\newline
	\begin{itemize}
		\item Einbau und Abänderung von Badezimmern und Saitäranlagen
		\item Loggienverglasungen
		\item Austausch von Fenstern und Fenstertüren in Schutzzonen
		\item sonstige Änderungen und Instandsetzungen von Bauwerken oder wesentliche Ä'nderungen von Äußeren Gestaltungen, Umwidmung, Stellplatzverplfichtungen
	\end{itemize}
\end{tcolorbox}

\begin{tcolorbox}[colback=lightgray!5!lightgray,colframe=lightgray!75!black]
text
\end{tcolorbox}
§62a Bewilligungsfreie Bauvorhaben
\begin{tcolorbox}[colback=lightgray!5!lightgray,colframe=lightgray!75!black]
Z.B.: \newline
	\begin{itemize}
		\item Abbruch von Bauwerken außerhalb Schutzzoine, Bausperrgebiet und Gebäude nach 01.01.1945
		\item Gartenhäuschen, Gerätehütten, Flugdächer
		\item Baustelleneinrichtungen
		\item Hauskanäle, Senkgruben und Hauskäranlagen
		\item Baustelleneinrichtungen
		\item Einfriedungen wenn nicht gegen Verkehrsflächen
		\item Fenstertausch
		\item Container für politische Zwecke in Wahlzeiten, Ausweichlokale, Einrichtung zur Versorgung
	\end{itemize}
\end{tcolorbox}

\begin{tcolorbox}[colback=lightgray!5!lightgray,colframe=lightgray!75!black]
test
\end{tcolorbox}

\begin{tcolorbox}[colback=lightgray!5!lightgray,colframe=lightgray!75!black]
ttse
\end{tcolorbox}

\begin{tcolorbox}[colback=lightgray!5!lightgray,colframe=lightgray!75!black]
ttesetr
\end{tcolorbox}

\begin{tcolorbox}[colback=lightgray!5!lightgray,colframe=lightgray!75!black]
teast
\end{tcolorbox}









